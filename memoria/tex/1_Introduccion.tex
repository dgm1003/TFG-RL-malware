\capitulo{1}{Introducción}

En la actualidad, cada vez existen más amenazas a los equipos informáticos, y las organizaciones tienen que invertir más recursos en proteger sus redes privadas y todos los dispositivos que se encuentran en ellas.

Para poder asegurarse de que su nivel de defensa es apropiado, la perspectiva del desarrollador que ha creado o configurado los servicios no siempre es suficiente, por lo que es común que las organizaciones contraten la ayuda de los llamados hackers éticos. Estos hackers éticos son individuos -- o equipos de individuos -- que se encargan de infiltrarse en una red, dispositivo o servicio, sin causar daño real, pero encontrando sus vulnerabilidades y puntos débiles. De este modo, al tener la perspectiva de alguien externo a la organización, y utilizar las técnicas y herramientas que un atacante real utilizaría, pueden encontrar áreas con seguridad insuficiente que podrían haber sido pasadas por alto.

Una situación que podría suceder en una organización es que alguna persona introdujese algún tipo de malware en alguno de los dispositivos de la red, y éste comenzase a propagarse por el resto de los equipos, posiblemente teniendo algún objetivo, como acceder a archivos confidenciales y enviarlos a alguna dirección externa. Ni la organización ni los hackers éticos podrán predecir qué objetivos tendrán los atacantes, ni qué métodos utilizarán, así que tendrán que anticiparse a todos los posibles, para así poder reducir el riesgo a valores aceptables.

Lo que se propone en este proyecto es el uso de aprendizaje automático para simular esta situación, la introducción de un malware a una red de ordenadores, y observar qué rutas tomaría para conseguir un objetivo con el menor riesgo. Sabiendo eso, se podrían introducir medidas de defensa para evitar que el malware tomase esas rutas, haciendo así más difícil el trabajo de los hackers maliciosos. La principal ventaja que ofrece este método es que puede encontrar rutas y puntos débiles que a un ser humano no se le habría ocurrido proteger, especialmente en redes grandes con gran cantidad de dispositivos.

Se ha optado por utilizar aprendizaje por refuerzo, pues al ser un modelo semi-supervisado no será necesario definir qué rutas se consideran exitosas para el malware y cuales no, sino que será suficiente con asignar valores de recompensa a las diferentes acciones que se pueden tomar. Además, es más robusto frente a variaciones en el entorno que otros modelos de aprendizaje, dando lugar a que, aunque cambie la situación de la red, el algoritmo pueda encontrar la ruta óptima igualmente.

Para facilitar su uso, este algoritmo de aprendizaje por refuerzo que se ha desarrollado también ha sido implementado dentro de una página web. 

\section{Estructura de la memoria}
La estructura de esta memoria es la siguiente:
\begin{itemize}
    \item \textbf{Introducción:} contiene una descripción del proyecto, además de una explicación de la estructura de la memoria y de los materiales complementarios.
    \item \textbf{Objetivos del proyecto:} listado de los objetivos, tanto generales como específicos.
    \item \textbf{Conceptos teóricos:} explicación de los principales conceptos teóricos necesarios para la comprensión del proyecto
    \item \textbf{Técnicas y herramientas:} descripción de las técnicas, metodologías y herramientas utilizadas, indicando por qué han sido elegidas frente a sus alternativas.
    \item \textbf{Aspectos relevantes del desarrollo del proyecto:} explicación de las diferentes decisiones tomadas durante el transcurso del proyecto, además de otras cuestiones que se consideren importantes.
    \item \textbf{Trabajos relacionados:} visión general de otros trabajos en el campo de la propagación de malware y aprendizaje automático en redes, y su relación al proyecto actual.
    \item \textbf{Conclusiones y líneas de trabajo futuras:} observaciones obtenidas después de haber completado el trabajo, y áreas en las que se puede profundizar o mejorar en caso de continuar el trabajo en el proyecto.
\end{itemize}

Además de la memoria, se cuenta con los siguientes apéndices:
\begin{description}
    \item[Apéndice A -] \textbf{Plan de proyecto software:} contiene tanto la planificación temporal del proyecto como los estudios de viabilidad realizados.
    \item[Apéndice B -] \textbf{Especificación de requisitos:} lista los objetivos, requisitos y casos de uso del trabajo.
    \item[Apéndice C -] \textbf{Especificación de diseño:} cubre las decisiones tomadas a la hora de diseñar los datos y procedimientos del sistema, además de una explicación detallada de su situación final.
    \item[Apéndice D -] \textbf{Documentación técnica de programación:} incluye todos los aspectos que se consideren relevantes para los programadores, desde la estructura de directorios o las instalaciones necesarias hasta las características especiales de los ficheros fuente.
    \item[Apéndice E -] \textbf{Documentación de usuario:} contiene un conjunto de explicaciones orientadas a los usuarios finales para que sean capaces de utilizar la aplicación correctamente y sin problemas.
\end{description}

\section{Materiales adicionales}

Junto con la memoria se proporciona un repositorio con el código fuente del algoritmo de aprendizaje por refuerzo, así como la página web lista para ser desplegada en cualquier ordenador para su uso. 

Dicho repositorio también se encuentra en el siguiente enlace: \url{https://github.com/dgm1003/TFG-RL-malware}

Se incluye también un archivo con la imagen Docker de la página web con el nombre \verb|tfg-website-image.tar|