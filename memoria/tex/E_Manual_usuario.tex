\apendice{Documentación de usuario}

\section{Introducción}
En este anexo se explicará los requisitos necesarios para ejecutar la aplicación y el proceso de instalación y puesta en marcha. También se incluye una guía para orientar al usuario por las diferentes páginas.

\section{Requisitos de usuario}
Los requisitos para poder utilizar la aplicación son los siguientes:

\begin{itemize}
    \item Tener un sistema operativo compatible con Docker\cite{docker:install}, como Ubuntu, o Windows con la versión Docker Desktop.
    \item Tener Docker instalado en dicho sistema operativo.
    \item Contar con un navegador de internet.
\end{itemize}


\section{Instalación y ejecución}

Para instalar y levantar la página web del proyecto, se deberán seguir los siguientes pasos:

\begin{enumerate}
    \item Descargar el repositorio del proyecto
    \item Abrir una ventana de la terminal
    \item Dirigirse a la carpeta del proyecto, y a continuación navegar hasta la carpeta \verb|/code/website| (por ejemplo, en Ubuntu y Windows se podrá navegar con el comando \verb|cd [nombre de directorio]| )
    \item Escribir \verb|./start.sh| si se está en un sistema operativo Ubuntu, o \verb|start.bat| si se está en un sistema operativo Windows.
    
    Este comando ejecutará el fichero del nombre correspondiente, que instalará la imagen Docker del proyecto, en caso de que no haya sido instalada anteriormente, y a continuación ejecutará un contenedor de dicha imagen. Si se está utilizando un sistema operativo que no soporte esas extensiones de fichero, en vez del último paso se puede abrir cualquiera de los ficheros en un editor de texto y copiar los dos comandos existentes dentro de la terminal.
    
    Deberán aparecer una o varias direcciones http por pantalla:
    
    \imagen{FlaskRunning}{Ejemplo de la aplicación en ejecución}
    
    \item introducir en el navegador web una de las url mostradas en el paso anterior
\end{enumerate}

Si se desea dejar de utilizar el programa, se puede detener con Ctrl+C.

\subsection{Importación de imagen Docker si se dispone de ella}

En caso de tener acceso a la imagen Docker en formato .tar, se podrá importar al equipo, evitando el proceso de instalación. Habrá que seguir los siguientes pasos:

\begin{enumerate}
    \item Descargar el repositorio del proyecto
    \item Abrir una ventana de la terminal
    \item Dirigirse a la carpeta que contenga la imagen (por ejemplo, en Ubuntu y Windows se podrá navegar con el comando\\ \verb|cd [nombre de directorio]| )
    \item Escribir el siguiente comando para importar la imagen docker:
    
        \verb|docker load --input [nombre del archivo .tar]|
    \item Escribir el siguiente comando para ejecutar el contenedor docker e iniciar la página web:
    
        \verb|docker run -it -p 8080:8080 [nombre de la imagen]|
    
    A partir de aquí, el procedimiento es idéntico a la otra alternativa. Deberán aparecer una o varias direcciones http por pantalla.
    
    \item introducir en el navegador web una de las url mostradas en el paso anterior
\end{enumerate}

Si se desea dejar de utilizar el programa, se puede detener con Ctrl+C.

\section{Manual del usuario}

A continuación se explicarán paso a paso las diferentes funcionalidades de la aplicación.

\subsection{Generación de red}

Al entrar en la página web, además de una breve descripción de la página, se le presenta al usuario con dos opciones para generar su red, brevemente explicadas:

\imagen{manualusuario1}{Página inicial, seleccionando opción de introducir datos de red}
\imagen{manualusuario2}{Página inicial, seleccionando opción de elegir red predefinida}

\subsubsection{Generación de red introduciendo valores}

Al seleccionar la primera opción, nos redirigirá al siguiente formulario:

\imagen{manualusuario3}{Página de introducir datos de la red}

En ella, nos pedirá los siguientes valores:

\begin{itemize}
    \item \underline{Número de nodos:} corresponde al número de dispositivos que tendrá la red. Es un campo obligatorio.
    \item \underline{Semilla (opcional):} las redes se generan de forma aleatoria, pero para poder controlar esta aleatoriedad se puede utilizar una semilla (también conocida como seed). Mientras dos redes tengan el resto de datos iguales, si tienen semillas diferentes sus generaciones serán diferentes, pero si su seed es la misma, serán idénticas, lo cual será muy útil si se quiere reproducir una ejecución varias veces con las mismas condiciones. En nuestro caso la semilla consistirá en un valor numérico.
    \item \underline{Porcentaje de nodos de alto riesgo (entre 0 y 1):} dentro de la red existen nodos que se considera que son de alto riesgo, pues sus defensas serán mayores y la probabilidad de detección será más alta. Por ello, el algoritmo intentará evitar esos nodos. En la generación de la red, se seleccionarán de forma aleatoria, y en este campo se puede definir la probabilidad de que cada nodo sea asignado un alto riesgo. Por defecto tiene un valor de 0,25, es decir, cada nodo tendrá un 25\% de probabilidades de ser de alto riesgo.
\end{itemize}

Una vez rellenados los campos, habrá que pulsar en el botón ``Siguiente'', y se pasará a la selección de nodo inicial y final:

\imagen{manualusuario4}{Página de selección de origen y meta}

Aparecerá una imagen, que mostrará los nodos de la red y todas las conexiones entre ellos. Cada nodo tendrá un identificador numérico. Debajo de la imagen, habrá dos campos, pidiendo el nodo inicial y objetivo respectivamente. En estos campos se tendrá que escribir el número identificativo de los nodos que se desee. 

También se podrá ver un tercer campo, opcional, para redefinir los nodos de riesgo. Se puede introducir una lista de números de nodos, separados por espacios, y estos nodos pasarán a ser los nuevos nodos de alto riesgo que el algoritmo deberá evitar a ser posible. Los antiguos nodos definidos como de alto riesgo volverán a ser nodos de riesgo normal, a no ser que se hayan incluido en la nueva lista.

Una vez se hayan rellenado los campos deseados, se deberá pulsar en el botón ``Siguiente'' para finalizar la configuración de la red.

\subsubsection{Generación de red seleccionando red predefinida}

Si se seleccionó la segunda opción en vez de la primera en la página principal, llevará a una página diferente, en la que aparecerá información sobre varios grafos, junto a una imagen representativa de cada uno.

\imagen{manualusuario5}{Página de selección de redes predefinidas}

El usuario simplemente tendrá que seleccionar la red que le interese y pulsar ``Siguiente'', sin tener que introducir ningún otro tipo de información.

\newpage
\subsection{Entrenamiento}

Independientemente de como se haya decidido generar la red, el sitio web acabará dirigiendo al usuario a la siguiente página:

\imagen{manualusuario6}{Página de configuración del entrenamiento}

En esta página se requieren los siguientes datos:

\begin{itemize}
    \item \underline{Tasa de aprendizaje (alpha):} esta tasa, entre 0 y 1, indicará la importancia que tendrán los movimientos nuevos durante el entrenamiento sobre el agente de aprendizaje por refuerzo. A tasas de aprendizaje muy bajas, su comportamiento se modificará a una velocidad más lenta, y tardará más en encontrar la solución óptima. Con tasas de aprendizaje muy elevadas, dará mucha importancia a los resultados de cada movimiento que realice, modificando su comportamiento muy rápidamente, y pudiendo pasarse por alto la solución óptima. Es por esto que es muy importante encontrar una tasa de aprendizaje apropiada a cada situación.
    
    \item \underline{Factor de descuento (gamma):} este valor, también entre 0 y 1, representa la importancia que le dará el agente a las posibles recompensas futuras. A valores muy altos, buscará maximizar las recompensas a corto plazo, y a valores muy bajos, intentará maximizar las recompensas a largo plazo. La recompensa a largo plazo es siempre una estimación, pues el agente no puede saber con certeza el valor exacto, por lo que puede darse la situación en la que, a factores de descuento muy bajos, el agente tome decisiones que aporten menor beneficio a corto plazo, esperando recompensas mayores en el futuro que no lleguen a materializarse. Por ello también es importante utilizar valores apropiados, y no muy extremos.
    
    \item \underline{Número de episodios del entrenamiento:} representa la duración del entrenamiento. A valores más altos tardará más tiempo en terminar, pero a valores muy bajos es posible que no le dé tiempo a encontrar la ruta óptima.
\end{itemize}

Una vez completados los campos, al pulsar en el botón ``Entrenar y obtener ruta'', la página hará exactamente eso, entrenará el agente de aprendizaje por refuerzo, y le pedirá buscar la ruta óptima entre el origen y el objetivo. Al finalizar, mostrará la página de resultados:

\imagen{manualusuario7}{Página de resultados}

En ella se podrá ver la ruta obtenida, tanto en la imagen como en el texto debajo de ella, además de la recompensa final, y un resumen de los datos de entrenamiento para facilitar la repetición del experimento.


\subsection{Repetición del proceso}

Si se quisiera volver a empezar todo el proceso, ya sea por algún error al introducir los datos, o para realizar otro entrenamiento después de haber finalizado uno, simplemente bastará con pulsar el botón de ``Volver a inicio'' presente en todas las páginas del sitio web.