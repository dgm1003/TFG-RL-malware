\apendice{Plan de Proyecto Software}

\section{Introducción}
Este apartado está dedicado a la planificación del proyecto, ya fuese realizada antes de o durante el proyecto. Se cubre la planificación temporal del proyecto y sus tareas, además de los estudios de viabilidad, tanto económica como legal.


\section{Planificación temporal}

Para la planificación temporal del proyecto se decidió basarse en la metodología Scrum\cite{Scrum}, con varias modificaciones - como el no usar roles debido al número limitado de participantes, o el número reducido de reuniones - que se explican más exhaustivamente en el capítulo 4 de la memoria, Técnicas y Herramientas.

Por lo tanto, el trabajo se dividió en varias iteraciones o Sprints, al final de cada cual se tendría una versión mejorada del proyecto. Se decidió que la duración de estos Sprints sería de dos semanas, aunque los últimos se acortaron a Sprints de una semana para poder tener un seguimiento más granular de las tareas finales. Entre cada sprint se realizaba una reunión de seguimiento, revisando las tareas del anterior y planificando el siguiente Sprint. 

%Podría meter una planificación general, a grandes rasgos antes de empezar. Línea temporal o gráfico Gantt o algo

Al añadir tareas al tablero Scrum, se les indicaba una estimación del trabajo necesario y su dificultad mediante un valor numérico. Dichos valores no eran equivalentes a una medida real concreta, pues, aunque se podría cuantificar el esfuerzo mediante horas de trabajo, no se ha considerado una medición apropiada, pues no representa bien la posible dificultad de la tarea (una tarea podría llevar mucho tiempo pero consistir simplemente en acciones repetitivas y conocidas, requiriendo más esfuerzo que otra tarea mucho más pequeña pero que necesitase un análisis exhaustivo y cuidadoso de los pasos a realizar). Por lo tanto, los valores asignados a las tareas son una medida relativa, indicando que una tarea con una estimación de un 1 requerirá la mitad de esfuerzo que una tarea con una estimación de un 2, y 8 veces menos que una tarea con una estimación de un 8.

Estos valores luego se utilizaban a la hora de planificar cada Sprint, viendo el valor acumulado de las tareas realizadas en los sprints anteriores, y seleccionando tareas de una dificultad similar (excepto en casos que hubiese que acelerar el ritmo).

En cuanto al uso de GitHub y ZenHub para la ejecución de dicha planificación, se utilizaron los siguientes elementos:

\begin{itemize}
    \item \textbf{Issues (GitHub):} representan cada una de las tareas a realizar. Tenían diferentes campos personalizables:
    \begin{itemize}
        \item \textbf{Labels (GitHub):} permiten clasificar las tareas según categoría (documentación, investigación, funcionalidades...)
        \item \textbf{Estimate/Story Points (ZenHub):} representan una estimación del trabajo necesario para terminar la tarea, y la dificultad de ella. Es una medida relativa, como se ha explicado anteriormente.
    \end{itemize}
    \item \textbf{Sprints (ZenHub), Milestones (GitHub):} equivalentes a los Sprints del método Scrum, agrupan las tareas según la iteración en la que se van a realizar. Son equivalentes, pero se han utilizado ambos pues cada uno ofrecía funcionalidades ligeramente diferentes: los Milestones permitían incluir descripciones sobre los objetivos de la iteración, y los Sprints permitían que una misma tarea perteneciese a varias iteraciones, en caso de que no se pudiese terminar en un solo sprint.
    \item \textbf{Projects (ZenHub), Epics (GitHub):} agrupan las tareas en diferentes objetivos a largo plazo, y pueden incluir tareas de Sprints diferentes que estén relacionadas con el mismo objetivo. En general, los proyectos cubren un mayor plazo de tiempo que las épicas. 
    \item \textbf{Pipelines (ZenHub):} representan las diferentes fases de progreso para las tareas. Son las siguientes:
    \begin{itemize}
        \item \textbf{New Issues:} Tareas que aún no se han etiquetado ni realizado la estimación del esfuerzo.
        \item \textbf{Epics:} Épicas, al no formar parte de sprints en concreto se dejan separadas del resto.
        \item \textbf{Icebox:} Tareas que no son urgentes
        \item \textbf{Optional Product Backlog:} Tareas que, si se cuenta con el tiempo y recursos suficientes, sería recomendable hacer, pero no son necesarias para finalizar el proyecto.
        \item \textbf{Product Backlog:} Tareas que tienen que ser realizadas, pero no han sido asignadas a ningún Sprint.
        \item \textbf{Sprint Backlog:} Tareas asignadas al Sprint actual, en las cuales no se ha empezado a trabajar.
        \item \textbf{In Progress:} Tareas en las que se está trabajando en el momento.
        \item \textbf{Review/QA:} Tareas que han sido terminadas por el alumno, pendientes de revisión con los tutores en la reunión de final de Sprint. 
        \item \textbf{Closed:} Tareas revisadas por los tutores y que han sido corregidas por el alumno en caso de que fuese necesario. Se dan por terminadas permanentemente. 
    \end{itemize}
\end{itemize}

El conjunto de tareas del proyecto se puede encontrar en el siguiente enlace:
\url{https://github.com/dgm1003/TFG-RL-malware/issues}

Debido al uso que se le ha otorgado a los dos últimos pipelines, a la hora de visualizar los burndown charts es necesario indicar que aquellas tareas en la pipeline \textit{Review/QA} o posterior se tienen que considerar como completadas.

\imagen{BurndownChartConfig}{Configuración para la visualización correcta de gráficos Burndown}

\clearpage
El conjunto de Sprints que se realizaron fue el siguiente:

\subsection{Sprint 1: 24/02/2022 - 09/03/2022}

\imagen{Sprint1}{Burndown report del Sprint 1}

En este sprint se seleccionó los programas a utilizar durante el transcurso de las prácticas, y se empezó a aprender sobre los conceptos relacionados con el trabajo y las librerías que sería más posible que se acabasen utilizando.
También se comenzó el trabajo en la memoria del proyecto, empezando a definir los objetivos de dicho proyecto.

Se definieron demasiadas tareas, y subestimó el esfuerzo necesario para realizar algunas de ellas, por lo que solo se acabó completando algo más de la mitad de las tareas asignadas a este Sprint. Algunas, como la búsqueda de artículos relacionados, o la visualización de una presentación sobre el aprendizaje por refuerzo, se decidión aplazar por razones como la falta de conocimientos del alumno, o que la presentación no se había transmitido aún cuando finalizó el sprint.
\newpage

\subsection{Sprint 2: 10/03/2022 - 23/03/2022}

\imagen{Sprint2}{Burndown report del Sprint 2}

En este sprint se continuó con el aprendizaje, empezando a codificar con varios ejemplos y tutoriales, tanto de Reinforcement Learning como de Deep Reinforcement Leraning. Además, se redactaron en la memoria los conceptos teóricos vistos hasta el momento, y se definieron los diferentes elementos del problema a diseñar.
\newpage

\subsection{Sprint 3: 24/03/2022 - 06/04/2022}

\imagen{Sprint3}{Burndown report del Sprint 3}

En este sprint se continuó con el aprendizaje, leyendo más sobre los conceptos de aprendizaje por refuerzo y realizando más tutoriales con diferentes aspectos que parecieron interesantes de aprender para aplicar posteriormente en el proyecto. Además, se definió el problema de forma concreta, para así poder programar una versión inicial del algoritmo que resolviese dicho problema para una red muy simplificada.

Debido a un aumento de la carga de trabajo de las asignaturas universitarias, en este sprint se cubrieron menos Story Points, pero por lo demás transcurrió sin problemas.
\newpage

\subsection{Sprint 4: 07/04/2022 - 20/04/2022}

\imagen{Sprint4}{Burndown report del Sprint 4}

En este sprint se realizaron unos últimos tutoriales que se consideraron relevantes, cerrando así la fase de aprendizaje de conceptos básicos. Además, se definió el formato de las redes para el problema, y se utilizó la librería NetworkX para la generación de dichas redes, probando varios métodos hasta encontrar el adecuado.

Por otro lado, se fueron añadiendo características al código, como la separación en clases y funciones, el uso de función de recompensa, o la separación del algoritmo en entorno y agente. Si se consideraba que dichas características aportaban alguna ventaja, se usaba esa nueva versión del código como base para el resto de modificaciones, cosa que ocurrió con todos los cambios propuestos en este sprint.
\newpage

\subsection{Sprint 5: 21/04/2022 - 04/05/2022}

\imagen{Sprint5}{Burndown report del Sprint 5}

En este sprint se continuó con la programación del algoritmo de Tablas Q, dejando completadas la mayoría de las características necesarias. Por otro lado, se realizó una planificación a grandes rasgos de las tareas necesarias a realizar para completar el proyecto, al haberse pasado ya el punto intermedio del tiempo disponible para la realización del TFG en primera convocatoria. También, relacionado con este último punto, se definieron posibles ramas por las que ampliar el problema una vez terminado el algoritmo de Tablas Q
\newpage

\subsection{Sprint 6: 05/05/2022 - 18/05/2022}

\imagen{Sprint6}{Burndown report del Sprint 6}

En este sprint se completó el código del algoritmo de aprendizaje por tablas Q, con todas las funcionalidades que se consideraron necesarias para el trabajo. 

Por otro lado, se intentó crear la página web, para que los usuarios pudiesen introducir los datos del programa y visualizar los resultados obtenidos de una forma intuitiva, sin necesidad de entender la forma en la que estuviera programado el código. Además, se pretendía avanzar con la memoria, pero debido a la alta carga de trabajo académico en esas fechas, motivos personales, y subestimar la cantidad de trabajo necesario para esas tareas, no se pudo completar ninguna de ellas antes de finalizar el sprint.

\newpage

\subsection{Sprint 7: 20/05/2022 - 31/05/2022}

\imagen{Sprint7}{Burndown report del Sprint 7}

En este sprint se tuvo como objetivo crear el frontend completo de la página web, además de continuar con la redacción de la memoria, y ambas tareas fueron completadas. 

También se empezó a crear el backend, realizando modificaciones al código para que se pudiese adaptar a las necesidades de la página web, pero no se pretendió acabar esta tarea en el sprint.

\subsection{Sprint 8: 02/06/2022 - 15/06/2022}

En este sprint se intentó implementar el backend de la página web, además de continuar con la memoria, pero se contó con menos tiempo del esperado, y surgieron varios problemas a la hora de crear el backend que no fueron solucionados dentro del sprint, por lo que no se completó ninguna tarea.
\newpage

\subsection{Sprint 9: 16/06/2022 - 22/06/2022}

\imagen{Sprint9}{Burndown report del Sprint 9}

En este sprint se continuó con las tareas planificadas para el sprint 8, especialmente la resolución de problemas relacionados con la página web, cosa que no se consiguió. Aún así, se pudo avanzar en la memoria.

\subsection{Sprint 10: 23/06/2022 - 30/06/2022}

En este sprint se continuó con la implementación de la página web. Después de probar varias alternativas y considerar otras soluciones, se decidió dejar de utilizar \verb|docker-compose| y \verb|setuptools| para configurar las imágenes de docker, y pasar a una estructura más simple. Con este cambio se resolvieron algunos de los problemas, pero no todos. No se pudo completar ningún sprint, pero sí hubo progreso en el proyecto.
\newpage

\subsection{Sprint 11: 31/06/2022 - 06/07/2022}

\imagen{Sprint11}{Burndown report del Sprint 11}

En este sprint se continuó con la resolución de problemas de la página web. Se consiguió que funcionase la aplicación de Flask dentro del contenedor Docker, y el resto de problemas eran ya del propio código dentro del backend y del frontend, los cuales se decidieron abordar más adelante en agosto. Por otro lado, se definieron los casos de prueba para el estudio de valores óptimos que se pretendía realizar en el siguiente sprint.

\subsection{Sprint 12: 07/07/2022 - 22/07/2022}

\imagen{Sprint12}{Burndown report del Sprint 12}

En este sprint se realizó un estudio sobre qué valores de entrenamiento resultaban óptimos para obtener buenos resultados en un tiempo aceptable. Para ello se tuvo que adaptar el algoritmo para poder recoger estadísticas, y para poder ejecutar un gran número de experimentos de forma consecutiva. Una vez creada la versión del algoritmo para el estudio, se ejecutaron las pruebas, y analizaron los resultados. Se quiso también avanzar en la memoria, pero debido a ciertos motivos personales no hubo tiempo de realizar esa tarea.

\subsection{Sprint 13: 02/08/2022 - 08/08/2022}

\imagen{Sprint13}{Burndown report del Sprint 13}

En este sprint no se pudo trabajar bastantes de los días, por lo que se hizo menos trabajo que de normal, pero se pudo avanzar en la memoria, completando otro capítulo y medio.
\newpage

\subsection{Sprint 14: 09/08/2022 - 22/08/2022}

\imagen{Sprint14}{Burndown report del Sprint 14}

En este sprint se trabajó principalmente en la página web, solucionando los problemas que se encontraron en julio, y terminando todas las funcionalidades planificadas inicialmente. Además, se continuó con el trabajo en la memoria.

\subsection{Sprint 15: 23/08/2022 - 04/09/2022}

\imagen{Sprint15}{Burndown report del Sprint 15}

Este sprint se dedicó completamente a escribir la memoria, pretendiendo finalizar la mayoría de los capítulos y la totalidad de los anexos. Este objetivo fue cumplido casi en su totalidad, quedando solo un capítulo y algunos apartados por acabar en el siguiente sprint.

\subsection{Sprint 16: 05/09/2022 - 11/09/2022}

\imagen{Sprint16}{Burndown report del Sprint 16}

En este sprint se quiso acabar la memoria, y solucionar los problemas encontrados en el código en sprints anteriores, además de realizar pequeñas mejoras a la página web para conseguir una experiencia de usuario más satisfactoria. Además, se preparó el repositorio para la realización de pruebas unitarias en el siguiente sprint.

Sin embargo, hubo dificultades a la hora de encontrar trabajos relacionados, por lo que ese apartado se tuvo que aplazar al siguiente sprint.
\newpage

\subsection{Sprint 17: 12/09/2022 - 18/09/2022}

\imagen{Sprint17}{Burndown report del Sprint 17}

En este sprint se finalizó el trabajo. Por un lado, se acabó la memoria y realizaron las correcciones necesarias. Por otro lado, se crearon los tests unitarios para asegurar la calidad del código, y realizaron las refactorizaciones necesarias. Por último, se añadieron funcionalidades a la página web y se revisó el funcionamiento de todos los elementos del proyecto.


\section{Estudio de viabilidad}

\subsection{Viabilidad económica}
En este apartado se estudiará el impacto económico que tendría este proyecto en caso de que se desarrollase con fines comerciales. Se contará con los siguientes tipos de costes:

\subsubsection{Costes de personal}
El trabajo ha sido realizado por un desarrollador, a lo largo de un periodo de 8 meses, en el que se considera que ha trabajado a jornada completa. Suponiendo un salario medio de 1200 euros netos mensuales por un desarrollador, teniendo en cuenta que hay que sumarle el impuesto de IRPF y la cotización de la seguridad social, se obtendría el siguiente coste total:

\begin{table}[h]
\centering
\begin{tabular}{| l | r |}
\hline
Concepto & Coste \\ \hline
Salario mensual neto & 1200,00€ \\
Retención IRPF (14\%) & 331,69€ \\
Seguridad Social (35,35\%) & 837,51€ \\
Salario mensual bruto & 2369,20€ \\
Salario anual bruto (12 pagas)   & 28.430,40€ \\ \hline
\textbf{Total 8 meses} & \textbf{18.953,60€} \\ \hline
\end{tabular} 
\caption{Costes de personal}
\end{table}


Para el cálculo de la cotización a la seguridad social, se han considerado los siguientes valores \cite{seg-social:cotizacion}:
\begin{itemize}
    \item Contingencias comunes: 28,30\% (23,60\% de parte de la empresa y 4,70\% de parte del trabajador)
    \item Desempleo: 7,05\% (5,50\% de parte de la empresa y 1,55\% de parte del trabajador)
    \item \textbf{Total:} 35,35\%
\end{itemize}

Para el cálculo del IRPF, se ha considerado un 14\% de retención para salarios brutos anuales de entre 20.200,00 y 35.200,00 euros en Castilla y León \cite{jcyl:irpf}.


\subsubsection{Costes de equipamiento}

En cuanto al equipamiento, solamente se ha utilizado un ordenador portátil con un coste alrededor de los 900 euros. Se ha hecho uso de un sistema operativo Windows 10 -- además de una máquina virtual con Ubuntu para probar la página web -- pero se podía haber utilizado Ubuntu a lo largo de todo el proceso de desarrollo. Al ser Ubuntu un sistema operativo de código abierto y sin coste, no se tendrán en cuenta costes de sistema operativo. El resto del software utilizado en el transcurso del trabajo también ha sido gratuito. 

Considerando una amortización del 26\% anual para el portátil, y teniendo en cuenta que ha sido utilizado durante 8 meses, los costes de equipamiento serán:

\begin{table}[h]
\centering
\begin{tabular}{| l | r | r | r |}
\hline
Concepto & Coste & Amortización (1 mes) & Amortización (8 meses) \\ \hline
Portátil & 900€ & 19.50€ & 156€ \\ \hline
\textbf{Total} & \textbf{900€} & \textbf{19.50€} & \textbf{156€} \\ \hline
\end{tabular} 
\caption{Costes de equipamiento}
\end{table}


\subsubsection{Costes de página web}

Para la presentación del trabajo fin de grado, la página se ha diseñado para que los usuarios la ejecuten de manera local en un contenedor Docker. Si se desease pasar a un modelo comercial, sería necesario obtener un dominio web, como por ejemplo, un dominio .com.

En este caso, el precio del registro de un dominio .com tiene una media de 12,95€, obteniendo el dominio durante un año. Para poder usar el dominio en años posteriores, sería necesario pagar una renovación, también de 12,95€ anuales. En este apartado se considerarán solamente los costes iniciales.


\begin{table}[h]
\centering
\begin{tabular}{| l | r |}
\hline
Concepto & Coste \\ \hline
Registro dominio & 12,95€ \\ \hline
\textbf{Total} & \textbf{12,95€} \\ \hline
\end{tabular} 
\caption{Costes de página web}
\end{table}

Si se desease aumentar el rendimiento del algoritmo, reduciendo los tiempos de espera en redes con mayor número de nodos, habría que invertir en servidores, pero no se considerará esta situación en este apartado.



\subsubsection{Coste Total}

El coste total del proyecto será el siguiente:


\begin{table}[h]
\centering
\begin{tabular}{| l | r |}
\hline
Concepto & Coste \\ \hline
Costes de personal & 18.953,60€ \\
Costes de equipamiento & 156,00€ \\
Costes de página web & 12,95€ \\ \hline
\textbf{Total} & \textbf{19.122,55€} \\ \hline
\end{tabular} 
\caption{Coste total}
\end{table}


\subsubsection{Posibles fuentes de ingreso}

Existen varias posibilidades de obtener beneficios con el proyecto. El primero sería añadir banners de publicidad a los laterales de la página web, pero por sí solo la página necesitaría un tráfico muy elevado para ofrecer beneficios suficientes para hacer rentable el proyecto.

Una alternativa mas viable sería ofrecer suscripciones para diferentes niveles de cuentas de usuario. A los usuarios gratuitos se les impondría un límite sobre el tamaño de las redes que podrían configurar, y el número de veces que podrían realizar entrenamientos cada cierto tiempo. A los usuarios de pago se les levantarían esas restricciones, pudiendo tener diferentes niveles con más funcionalidades. 

Se podría hasta combinar ambas ideas, mostrando publicidad a los usuarios gratuitos pero no a los usuarios de pago, incentivando así a los usuarios a mejorar sus cuentas.

\subsection{Viabilidad legal}
En este apartado se hablará del proceso de selección de licencia del proyecto, además de posibles problemas que pudiesen surgir de un mal uso del programa, y cómo evitarlos o remediarlos.


\subsubsection{Licencia}
Las licencias de las dependencias y librerías utilizadas en el proyecto son:

\begin{table}[h]
\centering
\begin{tabularx}{\textwidth}{| l | >{\raggedright\arraybackslash}X | l |}
\hline
\textbf{Dependencia}  & \textbf{Descripción} & \textbf{Licencia} \\ \hline
Flask & Framework de desarrollo web & BSD-3\cite{licenses:Flask} \\ \hline
Numpy & Biblioteca de operaciones matemáticas y manejo de matrices & BSD-3\cite{licenses:Numpy} \\  \hline
NetworkX & Biblioteca para manejo y análisis de grafos & BSD-3\cite{licenses:NetworkX} \\  \hline
Matplotlib & Biblioteca para generación de gráficas & PSF\cite{licenses:Matplotlib} \\ \hline
Unittest & Framework para automatización de pruebas unitarias & PSF\cite{licenses:Unittest} \\ \hline
Random & Biblioteca propia de Python para la generación de números y secuencias aleatorias & PSF\cite{licenses:Python} \\ \hline
\end{tabularx}
\caption{Dependencias del proyecto y sus licencias}
\label{tab:licencias}
\end{table}

Correspondiendo ``BSD-3'' con la licencia Berkeley Software Distribution de 3 cláusulas\cite{licenses:BSD}, y ``PSF'' con la Python Software Foundation License\cite{licenses:PSF}. 

Considerando las licencias de las dependencias, habría que utilizar una licencia compatible en el proyecto. Se estuvo debatiendo entre una licencia BSD de 3 cláusulas o una licencia GPL 3.0, y se acabó escogiendo la primera, ya que permitiría, en un futuro, crear un software basado en este proyecto con una licencia más restrictiva, pudiendo dejar de ser código abierto. 

Esto puede ser positivo en este caso concreto, debido a que, si se desarrollase más el código, podría ser utilizado con fines maliciosos, por lo que sería necesario imponer restricciones sobre su uso, como se explica en el siguiente apartado.

Por lo tanto, la licencia escogida es una licencia \textbf{BSD de 3 cláusulas}, o \textbf{"New BSD"}. Esta licencia permite el uso, redistribución y modificación del código siempre que se mantenga el copyright, las condiciones y el aviso legal en dichas redistribuciones. Además, se necesitará autorización escrita de parte de los creadores y colaboradores del código original en caso de querer utilizar sus nombres para promocionar obras derivadas.

\subsubsection{Consideraciones legales}
Debido a las restricciones impuestas actualmente sobre las redes que representa, sería difícil utilizar el proyecto con fines maliciosos. Si se decidiese ampliar el proyecto y quitar esas restricciones, lo primero que habría que hacer sería dejar de tener el código fuente abierto al público, y cambiar la licencia del proyecto, pues si no cualquier persona podría copiarlo y utilizar una versión que no controlemos para hacer lo que desee. Por otro lado, habría que añadir un sistema de gestión de usuarios, con logs de uso por cada usuario, para poder regular posibles situaciones sospechosas.

También se podría enviar un correo a INCIBE\cite{Incibe} pidiéndoles un análisis del riesgo potencial de la aplicación, para poder recibir consejo sobre cuestiones como a qué tipos de organizaciones se podría ofrecer el servicio.