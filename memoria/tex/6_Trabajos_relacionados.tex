\capitulo{6}{Trabajos relacionados}

En este capítulo se presentan varios trabajos realizados que comparten campo con el proyecto, en caso de que se desee profundizar más en los diferentes temas. 

Debido a las limitaciones impuestas sobre el alcance del proyecto, tanto sobre el entorno como sobre el algoritmo utilizado, y el contexto específico que se ha seleccionado, no se han encontrado trabajos similares, que permitiesen realizar comparativas entre sus resultados. Por eso, esta sección está enfocada con la idea de ofrecer diferentes puntos de partida, dado el caso de que se desee aprender más sobre las aplicaciones de la inteligencia artificial en el área de ciberseguridad, o sobre las estrategias de propagación de malware, entre otros asuntos.

Los artículos científicos que han sido seleccionados son los siguientes:

\subsection{Malware Detection and Prevention using Artificial Intelligence Techniques \cite{Articulo1}}

Este artículo tuvo como propósito la investigación sobre técnicas de Inteligencia Artificial (IA) para la detección de malware. Se realizó una revisión bibliográfica en busca de artículos científicos en varias bases de datos científicas, además de bases de datos on-line como IEEE o ScienceDirect, que tratasen sobre el tema del uso de inteligencia artificial en la detección y prevención de ataques malware. 

A continuación se explicaron las técnicas encontradas, hablando primero sobre las tres ramas de técnicas de detección de malware: signature-based (basado en firmas), anomaly-based (basado en anomalías) o heuristic-based (basado en heurísticas). Después, se pasa a explicar trabajos que utilizan inteligencia artificial para dicha detección, ya sea mediante redes convolucionales gráficas, mediante machine learning extrayendo características, o incluso utilizando máquinas virtuales para la detección de intrusiones. Por último, el artículo destaca las limitaciones que tienen estos sistemas propuestos.

Se ha considerado relevante este artículo, pues, al haber realizado una búsqueda sobre trabajos en este campo, y referenciándolos en el texto, sirve como lanzadera para investigación adicional.


\subsection{Impact of Network Structure on Malware Propagation: A Growth Curve Perspective \cite{Articulo2}}
En este trabajo, se toma un alcance diferente. En vez de buscar métodos o técnicas para detectar ataques de malware, el objetivo del trabajo fue analizar diferentes topologías de red para observar cuales eran más seguras frente a la propagación de un agente malware.

Para ello, se definieron primero las mecánicas de propagación de malware, indicando que el objetivo sería infectar al mayor número de dispositivos posible, pudiendo replicarse a si mismo el agente. Se explicaron las funciones y modelos que describían la propagación del malware. También se realizó un estudio previo de topologías de redes, diferenciando entre redes sociales -- en concreto MySpace -- y redes tecnológicas, es decir, redes propias de organizaciones.

Una vez hecho esto, se pasó a configurar y modelar las simulaciones de propagación y, con los resultados obtenidos, se creó un modelo de riesgo estructural, y se crearon estrategias de defensa basadas en dicho modelo de riesgo estructural. Al compararlas con otras estrategias de defensa, se observó como eran capaces de detener la propagación de una forma más rápida.

\subsection{Reinforcement learning based stochastic shortest path finding in wireless sensor networks \cite{Articulo3}}

En el siguiente artículo, se utiliza aprendizaje por refuerzo de Q-learning para encontrar rutas dentro de redes, al igual que en este proyecto, pero el artículo decide centrarse específicamente en redes de sensores inalámbricos, y su objetivo no tiene nada que ver con malware, sino que se enfoca en evitar retrasos en el tiempo y congestiones. Igualmente, se considera que las similitudes con el proyecto actual son lo suficientemente relevantes como para incluirse.

En el artículo, se propone un modelo estocástico, en el cual la longitud de los enlaces entre nodos va definida por una función de probabilidad. De este modo, no se puede saber con claridad cual será la recompensa por moverse a través de un enlace. Se diseñan algoritmos de Q-learning y SARSA para la navegación por este entorno, y se ejecutan. A continuación, se compara su eficiencia con otros algoritmos de enrutamiento de paquetes, midiendo el valor de regret, que consiste en la diferencia entre la recompensa obtenida y la recompensa óptima teórica.

\subsection{Advanced malware propagation on random complex networks \cite{Articulo4}}

Se incluye también un último trabajo, que propone el uso de autómatas celulares para la propagación de malware dentro de redes complejas aleatorias. Primero, se diseña el autómata celular, y a continuación se definen las redes, indicando que tendrán un número fijo de dispositivos, y estos dispositivos tendrán cuatro posibles estados: susceptible, infectado, atacado y recuperado. Después de esto se realiza un estudio de la efectividad del autómata dentro de redes aleatorias complejas, variando diferentes características del entorno, como los valores de probabilidad utilizados en la generación de la red, o la selección de nodos iniciales.

