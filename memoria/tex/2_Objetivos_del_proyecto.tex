\capitulo{2}{Objetivos del proyecto}

Este proyecto cuenta con varios objetivos: 

\section{Objetivos Generales:}

\begin{itemize}
    \item Diseñar y desarrollar un algoritmo de aprendizaje por refuerzo que simule la propagación de un malware por una red de ordenadores, buscando la ruta óptima para llegar a un equipo objetivo.
    \item Crear una interfaz gráfica que permita a un usuario con conocimientos mínimos de programación configurar el algoritmo y utilizarlo.
\end{itemize}

 
 \section{Objetivos Específicos:}
 
 
 \begin{enumerate}
     
     \item Obtener conocimientos sobre aprendizaje por refuerzo, a través de documentación técnica y de investigación, y aplicar dichos conocimientos en el trabajo, centrándose especialmente en los Procesos de Decisión de Markov (MDP) y sus elementos: estados, acciones y recompensas.
    
     \item Facilitar un entorno que permita representar mediante un grafo los datos de una red de ordenadores, que sea adaptable a diferentes topologías, sea capaz de representar los posibles estados de la red, y tenga definidas las diferentes acciones que puede tomar el agente malware introducido en la red, con sus recompensas asociadas.
     
     \item Ofrecer una forma de visualizar el estado del entorno, representando tanto la topología de la red como el recorrido tomado por el malware y los dispositivos que han sido infectados.
     
     \item Incluir también un agente de aprendizaje por refuerzo que trabaje sobre dicho entorno para entrenar la propagación del malware en la red, centrándose en la infección de un nodo objetivo.
     
     \item Proporcionar una interfaz, como puede ser una página web, que permita a un usuario trabajar con el entorno y agente existentes, personalizándolos a sus necesidades, y obteniendo los resultados del entrenamiento y búsqueda de ruta consiguientes.
     
     \item Realizar diferentes pruebas y estudios sobre el algoritmo creado para demostrar su eficiencia y adaptabilidad.
     
 \end{enumerate}
