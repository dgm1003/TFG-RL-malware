\apendice{Especificación de Requisitos}

\section{Introducción}
En este anexo se tratan los objetivos que se pretenden conseguir con este proyecto, los requisitos funcionales que se han fijado para cubrir dichos objetivos, y los casos de uso relacionados con ellos.

\section{Objetivos generales}

\begin{itemize}
    \item Realizar un algoritmo de aprendizaje por refuerzo que simule la propagación de un malware por una red de ordenadores, buscando la ruta óptima para llegar a un equipo objetivo.
    \item Crear una interfaz gráfica que permita a un usuario con conocimientos mínimos de programación configurar el algoritmo y utilizarlo.
\end{itemize}

\section{Catalogo de requisitos}
A continuación se listan los requisitos que se han definido para el proyecto.

\begin{description}
    \item[RF 1:] El usuario debe ser capaz de configurar y ejecutar un programa que permita entrenar un algoritmo de aprendizaje por refuerzo.
    \begin{description}
        \item[RF 1.1:] El programa deberá admitir un entorno sobre el que realizar el entrenamiento.
        \item[RF 1.2:] El usuario debe poder definir las tasas de aprendizaje y de descuento, y el número de episodios o iteraciones del entrenamiento.
        \item[RF 1.3:] Una vez introducidas las variables y definido el entorno, el usuario deberá ser capaz de entrenar el algoritmo.
        \item[RF 1.4:] El programa deberá poder encontrar la ruta óptima entre el punto inicial y la meta, una vez entrenado.
    \end{description}
    \item[RF 2:] El usuario debe ser capaz de generar y manejar un entorno que represente un malware en una red de ordenadores.
    \begin{description}
        \item[RF 2.1:] El entorno deberá contener una representación de la red de ordenadores, con información de qué dispositivos están infectados o son de alto riesgo.
        \item[RF 2.2:] El usuario debe poder definir las características de dicha red, que serán el número de dispositivos de la red, el porcentaje de dispositivos de alto riesgo, el dispositivo en el que empezará situado el malware, y el dispositivo objetivo que se quiere infectar. También podrá definir la seed utilizada en operaciones aleatorias, para poder reproducir pruebas de manera exacta.
        \item[RF 2.3:] El usuario debe poder seleccionar de entre varias redes predefinidas, con diferentes características, en vez de definir todos sus valores.
        \item[RF 2.4:] El entorno debe poder mostrar un listado de posibles acciones a realizar, dado su estado actual.
        \item[RF 2.5:] El usuario debe poder indicar al entorno que realice una acción, recibiendo información sobre el estado en el que se encuentra el entorno después de realizar dicha acción, la recompensa que ha obtenido, y si ha completado su objetivo o no.
        \item[RF 2.6:] El entorno debe ser capaz de mostrar una representación gráfica de su estado actual.
        \item[RF 2.7:] El usuario deberá poder reestablecer los valores del entorno a sus estados iniciales, para así poder reutilizar dicho entorno en otras situaciones.
    \end{description}
    \item[RF 3:] El usuario debe poder acceder a una página web que permita una fácil utilización del agente y entorno programados, sin necesidad de conocimientos de programación.
    \begin{description}
        \item[RF 3.1:] El usuario debe ser capaz de obtener una red de acuerdo a sus necesidades, ya sea introduciendo los valores de los parámetros concretos en un formulario, o seleccionando botones que correspondan a cada una de las redes predefinidas.
        \item[RF 3.2:] En caso de introducir manualmente los parámetros de la red, el usuario deberá poder seleccionar el nodo de origen y el nodo meta, después de ver una imagen de la red generada.
        \item[RF 3.3:] El usuario debe tener la posibilidad de definir mediante un formulario las tasas de aprendizaje y de descuento, y el número de episodios del entrenamiento, una vez terminada de configurar la red. 
        \item[RF 3.4:] Una vez definidos todos los parámetros, la página deberá pasar al entrenamiento del algoritmo, y una vez entrenado a la búsqueda de la ruta óptima desde el inicio hasta la meta, mostrando al final de este proceso la ruta y recompensa obtenida, además de una representación gráfica de la ruta tomada dentro de la red.
        \item[RF 3.5:] El usuario debe tener la opción de repetir la selección de red y el entrenamiento de agente tantas veces como desee.
    \end{description}
\end{description}
\newpage

\section{Especificación de requisitos}
\subsection{Diagrama de casos de uso}
A continuación se muestra el diagrama de casos de uso de la página web creada, pues es la parte del proyecto a la que accederá un usuario final, y contiene las funcionalidades del resto de componentes.

\imagen{DiagramaCasosDeUso}{Diagrama de casos de uso}

\subsection{Actores}
En el sistema actual solo existirá un actor, el usuario final. En caso de añadir algunas de las características comentadas en el estudio de viabilidad, sería necesario incluir como segundo actor un administrador de usuarios, con nuevos casos de uso asociados.

\newpage
\subsection{Especificación de casos de uso}
\begin{table}[h]
\centering
\begin{tabularx}{\textwidth}{| l | >{\raggedright\arraybackslash}X |}
\hline
\textbf{CU-01} & \textbf{Configurar red} \\ \hline
Descripción     & Permite al usuario configurar la red como desee. \\ \hline
Requisitos      & RF-2, RF-2.1, RF-2.2, RF-2.3, RF-2.6, RF-3, RF-3.1, RF-3.2 \\ \hline
Precondiciones  & La página web está disponible. \\ \hline
Secuencia       & 1. El usuario entra en la página web. \\
                & 2. El usuario selecciona si quiere introducir datos manualmente o seleccionar una red predefinida. \\
                & 3. El usuario completa los formularios de la opción que haya seleccionado. \\
                & 4. El usuario pulsa el botón ``Siguiente''. \\
                & 5. Si los datos son válidos, se crea una nueva red. \\ \hline
Postcondiciones & Una red con las especificaciones introducidas es creada. \\ \hline
Excepciones     & Se introducen los datos de forma incorrecta. \\ \hline
Importancia     & Alta \\ \hline
\end{tabularx}
\caption{Caso de uso 1: Configurar red}
\label{tab:CU-01}
\end{table}


\begin{table}[h]
\centering
\begin{tabularx}{\textwidth}{| l | >{\raggedright\arraybackslash}X |}
\hline
\textbf{CU-02}  & \textbf{Obtener ruta óptima} \\ \hline
Descripción     & Permite entrenar el algoritmo de aprendizaje por refuerzo y encontrar la ruta óptima del inicio a la meta. \\ \hline
Requisitos      & RF-1, RF-1.1, RF-1.2, RF-1.3, RF-1.4, RF-2, RF-2.4, RF-2.5, RF-2.6, RF-3, RF-3.3, RF-3.4, RF-3.5 \\ \hline
Precondiciones  & Se ha creado una red \\ \hline
Secuencia       & 1. El usuario introduce los datos del entrenamiento. \\
                & 2. El usuario pulsa el botón ``Entrenar''. \\
                & 3. Se crea un agente con la red existente. \\
                & 4. Se entrena el agente con los parámetros introducidos. \\
                & 5. Se pide al agente encontrar la ruta óptima. \\
                & 6. Se muestran los resultados. \\ \hline
Postcondiciones & Se muestra la ruta encontrada tanto en formato de texto como en representación gráfica. \\ \hline
Excepciones     & Los datos de entrenamiento introducidos son inválidos. \\ \hline
Importancia     & Alta \\ \hline
\end{tabularx}
\caption{Caso de uso 2: Obtener ruta óptima}
\label{tab:CU-02}
\end{table}


\begin{table}[h]
\centering
\begin{tabularx}{\textwidth}{| l | >{\raggedright\arraybackslash}X |}
\hline
\textbf{CU-03}  & \textbf{Introducir datos de la red} \\ \hline
Descripción     & Permite al usuario introducir los datos necesarios para generar una red. \\ \hline
Requisitos      & RF-2, RF-2.1, RF-2.2 \\ \hline
Precondiciones  & La página web está disponible. \\ \hline
Secuencia       & 1. El usuario selecciona el botón ``Introducir valores''. \\
                & 2. El usuario introduce el número de nodos (mayor que 0). \\
                & 3. Opcionalmente, el usuario introduce un valor numérico que sirva como seed para la generación aleatoria. \\
                & 4. El usuario introduce el porcentaje de nodos que serán de alto. riesgo (como un decimal entre 0 y 1) \\
                & 5. El usuario pulsa en ``Siguiente''. \\ \hline
Postcondiciones & Se ha creado una red sin nodo de inicio o meta. \\ \hline
Excepciones     & El número de nodos introducido es menor que 0, la seed introducida no es un valor numérico o el valor del porcentaje está por debajo de 0 o por encima de 1. \\ \hline
Importancia     & Alta \\ \hline
\end{tabularx}
\caption{Caso de uso 3: Introducir datos de la red}
\label{tab:CU-03}
\end{table}


\begin{table}[h]
\centering
\begin{tabularx}{\textwidth}{| l | >{\raggedright\arraybackslash}X |}
\hline
\textbf{CU-04}  & \textbf{Seleccionar inicio y meta} \\ \hline
Descripción     & Permite al usuario decidir qué nodos serán el origen y el objetivo del entrenamiento. \\ \hline
Requisitos      & RF-2, RF-2.1, RF-2.2, RF-2.6, RF-3, RF-3.2 \\ \hline
Precondiciones  & Se ha creado una red anteriormente por el método de introducir parámetros. \\ \hline
Secuencia       & 1. Se crea una representación gráfica de la red existente y se muestra al usuario. \\
                & 2. El usuario introduce el número del nodo inicial (no podrá ser menor a 0 o mayor o igual que el número de nodos de la red) \\
                & 3. El usuario introduce el número del nodo destino (no podrá ser menor a 0 o mayor o igual que el número de nodos de la red) \\
                & 4. El usuario pulsa en ``Siguiente''. \\
                & 5. Se actualiza la red con la nueva información. \\ \hline
Postcondiciones & La red actual pasa a tener nodos inicial y meta definidos. \\ \hline
Excepciones     & Los números de nodo inicial o meta introducidos son inválidos. \\ \hline
Importancia     & Alta \\ \hline
\end{tabularx}
\caption{Caso de uso 4: Seleccionar inicio y meta}
\label{tab:CU-04}
\end{table}


\begin{table}[h]
\centering
\begin{tabularx}{\textwidth}{| l | >{\raggedright\arraybackslash}X |}
\hline
\textbf{CU-05}  & \textbf{Seleccionar red predefinida} \\ \hline
Descripción     & Permite al usuario seleccionar de entre varias redes con diferentes configuraciones sin tener que introducir los datos manualmente. \\ \hline
Requisitos      & RF-2, RF-2.1, RF-2.3, RF-2.6 RF-3, RF-3.1 \\ \hline
Precondiciones  & La página web está disponible. \\ \hline
Secuencia       & 1. El usuario pulsa ``Seleccionar red predefinida''. \\
                & 2. Se muestran varios botones con información sobre las configuraciones de cada una de las redes, y una imagen de cada una. \\
                & 3. El usuario selecciona la red que desee utilizar. \\
                & 4. El usuario pulsa ``Siguiente''. \\
                & 5. Se crea la red con los parámetros seleccionados. \\ \hline
Postcondiciones & Una red con las especificaciones introducidas es creada. \\ \hline
Excepciones     & El usuario no selecciona ninguna red. \\ \hline
Importancia     & Media \\ \hline
\end{tabularx}
\caption{Caso de uso 5: Seleccionar red predefinida}
\label{tab:CU-05}
\end{table}


\begin{table}[h]
\centering
\begin{tabularx}{\textwidth}{| l | >{\raggedright\arraybackslash}X |}
\hline
\textbf{CU-06}  & \textbf{Configurar entrenamiento} \\ \hline
Descripción     & Permite al usuario definir los parámetros del entrenamiento. \\ \hline
Requisitos      & RF-1, RF-1.2, RF-3, RF-3.3 \\ \hline
Precondiciones  & Existe una red creada anteriormente. \\ \hline
Secuencia       & 1. El usuario introduce la tasa de aprendizaje (entre 0 y 1) \\
                & 2. El usuario introduce el factor de descuento (entre 0 y 1) \\
                & 3. El usuario introduce el número de episodios del entrenamiento (mayor que 0) \\
                & 4. El usuario pulsa el botón ``Entrenar''. \\
                & 5. Se guardan los datos del entrenamiento. \\ \hline
Postcondiciones & Los datos del entrenamiento han sido guardados. \\ \hline
Excepciones     & Se introducen valores inválidos en alguno de los tres campos. \\ \hline
Importancia     & Alta \\ \hline
\end{tabularx}
\caption{Caso de uso 6: Configurar entrenamiento}
\label{tab:CU-06}
\end{table}


\begin{table}[h]
\centering
\begin{tabularx}{\textwidth}{| l | >{\raggedright\arraybackslash}X |}
\hline
\textbf{CU-07}  & \textbf{Entrenar algoritmo y buscar ruta} \\ \hline
Descripción     & Permite al usuario obtener la ruta óptima entre el inicio y la meta después de un entrenamiento del algoritmo de aprendizaje por refuerzo. \\ \hline
Requisitos      & RF-1, RF-1.3, RF-1.4, RF-2, RF-2.4, RF-2.5, RF-3, RF-3.4 \\ \hline
Precondiciones  & Existe una red creada, y se tienen los parámetros del entrenamiento guardados. \\ \hline
Secuencia       & 1. Se crea un agente con la red existente. \\
                & 2. Se entrena el agente con los parámetros introducidos. \\
                & 3. Se pide al agente encontrar la ruta óptima. \\
                & 4. Se guarda la ruta encontrada y recompensa obtenida. \\ \hline
Postcondiciones & La ruta y recompensa obtenidas han sido guardadas. \\ \hline
Excepciones     & Ninguna \\ \hline
Importancia     & Alta \\ \hline
\end{tabularx}
\caption{Caso de uso 7: Entrenar algoritmo y buscar ruta}
\label{tab:CU-07}
\end{table}


\begin{table}[h]
\centering
\begin{tabularx}{\textwidth}{| l | >{\raggedright\arraybackslash}X |}
\hline
\textbf{CU-08}  & \textbf{Obtener resumen de resultados} \\ \hline
Descripción     & Permite al usuario visualizar los resultados del entrenamiento. \\ \hline
Requisitos      & RF-2, RF-2.6, RF-3, RF-3.4 \\ \hline
Precondiciones  & Se ha creado una red y se tiene una ruta desde el inicio hasta la meta. \\ \hline
Secuencia       & 1. Se genera una representación gráfica del recorrido de la ruta sobre la red. \\
                & 2. Se muestra la imagen generada junto a un resumen de los datos del entrenamiento y los resultados obtenidos. \\ \hline
Postcondiciones & Ninguna \\ \hline
Excepciones     & Ninguna \\ \hline
Importancia     & Alta \\ \hline
\end{tabularx}
\caption{Caso de uso 8: Obtener resumen de resultados}
\label{tab:CU-08}
\end{table}


\begin{table}[h]
\centering
\begin{tabularx}{\textwidth}{| l | >{\raggedright\arraybackslash}X |}
\hline
\textbf{CU-09}  & \textbf{Generar visualización del entorno} \\ \hline
Descripción     & Permite a la página web generar una visualización del estado en un momento concreto de la red de ordenadores. \\ \hline
Requisitos      & RF-2, RF-2.6 \\ \hline
Precondiciones  & Existe una red. \\ \hline
Secuencia       & 1. Se pide al entorno generar una representación gráfica de la red. \\
                & 2. En caso de haber una ruta, se pide al entorno priorizar los nodos de la ruta para que se vean más claramente que el resto. \\
                & 3. Se pide al entorno guardar la representación generada con un nombre concreto. \\ \hline
Postcondiciones & Se ha guardado una visualización de la red. \\ \hline
Excepciones     & Ninguna \\ \hline
Importancia     & Alta \\ \hline
\end{tabularx}
\caption{Caso de uso 9: Generar visualización del entorno}
\label{tab:CU-09}
\end{table}

