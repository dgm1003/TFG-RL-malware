\capitulo{7}{Conclusiones y Líneas de trabajo futuras}

En este apartado se exponen observaciones y conclusiones obtenidas al completar el trabajo, además de las áreas por las que se puede continuar el desarrollo en caso de seguir con el trabajo en el proyecto.

\section{Conclusiones}
Al finalizar el proyecto, hemos llegado a las siguientes conclusiones:

Por un lado, la mayoría de los objetivos que se definieron inicialmente han sido cumplidos satisfactoriamente. Se ha creado un entorno que representa una red de ordenadores, y el malware localizado en ella, basándose en las características definidas como necesarias para una simulación correcta. También se ha programado un agente que actúa sobre dicho entorno utilizando aprendizaje por refuerzo para encontrar la ruta óptima hasta un nodo objetivo. Además se ha diseñado una página web que sirve como interfaz para la fácil utilización de dicho agente.

Sin embargo, el objetivo de realizar pruebas y estudios sobre el funcionamiento del algoritmo no se ha podido llegar a realizar por completo, debido principalmente a una falta de tiempo. Se realizó un estudio de convergencia, pero inicialmente se pretendió también examinar otras capacidades del algoritmo, como la robustez frente a cambios en el entorno.

Este proyecto ha sido muy útil a la hora de aplicar una gran cantidad de los conocimientos obtenidos a lo largo del curso universitario, a la vez que se iban adquiriendo más conocimientos concretos para el trabajo. Se han podido aplicar conocimientos de metodologías ágiles, programación estructurada, pruebas de código o diseño de páginas web, entre otras cosas. Para algunas de las áreas, como el uso de contenedores Docker, se han ampliado los conocimientos adquiridos en el grado para poder ajustarse a las necesidades del proyecto. 

Por otro lado, el realizar este proyecto me ha ayudado a tener una primera toma de contacto con el desarrollo de trabajos a largo plazo, desde su planificación hasta su desarrollo y consecuente mejora. Ha servido para obtener experiencia a la hora de estimar el esfuerzo necesario en las diferentes tareas que formaban parte del trabajo, y el esfuerzo que se era capaz de realizar en los diferentes plazos de tiempo. Esto era algo con lo que se ha tenido dificultad, pero se ha acabado con un mayor conocimiento tanto del proceso de planificación de proyectos como de los límites de uno mismo.

No obstante, se habría preferido incluir más elementos al proyecto, y extenderlo algo más, pero no ha sido posible dentro del tiempo del que se disponía. Estas ampliaciones se exponen en el siguiente apartado.

\section{Líneas de trabajo futuras}

A la hora de ampliar el alcance del proyecto, e incluir funciones adicionales, se sugieren las siguientes avenidas de mejora:

\begin{itemize}
    \item \textbf{Adición de fuentes de incertidumbre:} Para mostrar la adaptabilidad de los algoritmos de aprendizaje por refuerzo, sería una buena idea añadir fuentes de ruido o incertidumbre durante el entrenamiento del agente, como la variación de la peligrosidad de los nodos o la modificación de conexiones entre ellos.
    \item \textbf{Estudio de adaptabilidad:} Después de realizar el punto anterior, sería interesante realizar un estudio de la capacidad de adaptarse que tendría el agente programado, observando su respuesta a diferentes niveles de ruido, y si sería capaz de ajustarse apropiadamente.
    \item \textbf{Ampliación de los objetivos:} El objetivo actual, de infección de un único nodo concreto, es útil a la hora de proteger ese dispositivo, pero la mayoría de malware suelen tener objetivos más complejos. Por lo tanto, se propone la necesidad de infectar múltiples dispositivos objetivo, o la posibilidad de infectar equipos a lo largo del camino para reducir su riesgo en caso de tener que pasar otra vez por ellos, o ampliaciones similares al objetivo del malware.
    \item \textbf{Publicación de la página web:} Actualmente, la página se encuentra dentro de un contenedor Docker, requiriendo su descarga y ejecución en un dispositivo. Una posibilidad para que el proyecto sea accesible a más gente sería la publicación de la página en internet, registrando un dominio para que se pudiese acceder desde cualquier navegador. Junto a esto, se podrían incluir mejoras a la propia página, como una gestión de usuarios, mayor libertad a la hora de generar las redes, u otras formas de representar visualmente los resultados del entrenamiento.
    \item \textbf{Adición de complejidad al entorno:} el entorno presente en el proyecto considera todos los dispositivos conectados a la red como iguales, diferenciándose únicamente por su riesgo, y todas las conexiones iguales también. Se podría ampliar la complejidad de esta red, añadiendo diferentes características a los equipos, como switches o firewalls, y a las conexiones, que afecten al comportamiento del malware.
    \item \textbf{Algoritmo de Deep Reinforcement Learning:} Por último, se podría ir un paso más allá, y crear un nuevo agente que, en vez de Q-Learning, utilice aprendizaje profundo por refuerzo y redes neuronales para realizar el entrenamiento y búsqueda de rutas. Una vez creado, se podrían realizar estudios comparando ambos agentes para observar sus diferencias, y cuál obtiene mejores resultados.
\end{itemize}